\chapter*{Introduction générale}
\markboth{Introduction générale}{} %pour afficher l'entête
\addcontentsline{toc}{chapter}{Introduction générale}

\setlength{\parskip}{1em}
\setlength{\parindent}{1cm}

\large {
L'intégration des objets connectés dans les systèmes d’information des entreprises est devenue une nécessité pour répondre aux exigences accrues du marché et à l’évolution incessante de la technologie \cite{antoine2019vers}. Dans ce contexte, ce rapport de Mémoire de Mastère porte sur le déploiement de l'infrastructure IT, le développement d'un système d'information et l'intégration des objets connectés au système d’information de l’entreprise Zeta Engineering.

Ce rapport est structuré en quatre chapitres pour expliquer notre approche et notre travail :

\begin{itemize}
\item Le premier chapitre présente le projet, son contexte, les outils utilisées, ainsi que l'organisme d'accueil. Nous y exposons les activités de l'entreprise, ses objectifs et ses besoins en matière d'infrastructure.
\item Le deuxième chapitre décrit le déploiement de l'infrastructure IT. Nous détaillons les défis rencontré et les solutions proposées.
\item Le troisième chapitre explique le déploiement du système d'information. Nous présentons les choix technologiques que nous avons effectués, les différentes étapes de la mise en place du système.
\item Le dernier chapitre se concentre sur l'intégration des objets connectés. Nous expliquons comment nous avons relié les objets non connectés au système d'information pour collecter et analyser les données. De plus, nous discutons des défis rencontrés et des solutions mises en place pour les surmonter.
\end{itemize}

En conclusion de cette introduction générale, nous avons souligné l'importance croissante de l'intégration des objets connectés dans les systèmes d’information des entreprises. Ce rapport mettra en lumière notre expérience avec l'entreprise Zeta Engineering en détaillant les étapes clés de notre projet.
}
