\chapter*{Introduction générale}
\markboth{Introduction générale}{} %pour afficher l'entete
\addcontentsline{toc}{chapter}{Introduction générale}
\large {

\setlength{\parskip}{1em}
\setlength{\parindent}{1cm}

L'intégration des objets connectés dans les systèmes d’information des entreprises est une nécessité pour répondre aux exigences accrues du marché et à l’évolution incessante de la technologie \cite{antoine2019vers}. Dans ce contexte, ce rapport de Mémoire de Mastère porte sur l'intégration des objets connectés au système d’information de l’entreprise Zeta Engineering. 


Les objets connectés ont connu un développement sans précédent ces dernières années. Leur intégration dans les systèmes d'information des entreprises offre des opportunités considérables pour améliorer l'efficacité opérationnelle et la prise de décision.

Ce rapport se concentre sur notre expérience de travail avec l'organisme d'accueil pour réaliser ce projet. Nous travaillons en étroite collaboration avec l'équipe IT de cette entreprise pour déployer l'infrastructure nécessaire, intégrer les objets connectés et développer un système d'information approprié pour gérer les données collectées. Nous avons divisé notre rapport en quatre chapitres pour expliquer notre approche et notre travail :

\begin{itemize}
  \item Le premier chapitre présente le projet et son contexte, ainsi que l'organisme d'accueil. Nous expliquons les activités de l'entreprise, ses objectifs et ses besoins en matière d'objets connectés. De même nous présentons également les parties prenantes impliquées dans ce projet, afin de mieux comprendre les enjeux et les attentes de chacun. Cette présentation, nous permettons d'établir les fondements nécessaires pour la mise en place d'une infrastructure et d'un système d'information adaptés à l'intégration des objets connectés. \\
  \item Le deuxième chapitre décrit le déploiement de l'infrastructure IT. Nous détaillons les différents éléments nécessaires à intégrer dans une entreprise comme les serveurs et les réseaux. \\
  \item Le troisième chapitre explique le déploiement du système d'information. Nous présentons les choix technologiques que nous avons faits, les différentes étapes de la mise en place du système et les tests effectués pour valider son fonctionnement. \\
  \item Le dernier chapitre, se concentre sur l'intégration des objets connectés. Nous expliquons comment nous avons relié les objets connectés au système d'information pour collecter et analyser les données collectées. Nous discutons également des défis rencontrés et des solutions mises en place pour y faire face. \\
\end{itemize}

En conclusion de cette introduction générale, nous avons présenté l'importance croissante de l'intégration des objets connectés dans les systèmes d’information des entreprises. Ce rapport se concentrera sur notre expérience avec l'entreprise Zeta Engineering, détaillant les étapes clés de notre projet, du déploiement de l'infrastructure au développement du système d'information, en passant par le déploiement de l'infrastructure IT et SI puis l'intégration  des objets connectés.



}